%%%%%%%%%%%%%%%%%%%%%%%%%%%%%%%%%%%%%%%%%
% Medium Length Professional CV
% LaTeX Template
% Version 2.0 (8/5/13)
%
% This template has been downloaded from:
% http://www.LaTeXTemplates.com
%
% Original author:
% Trey Hunner (http://www.treyhunner.com/)
%
% Important note:
% This template requires the resume.cls file to be in the same directory as the
% .tex file. The resume.cls file provides the resume style used for structuring the
% document.
%
%%%%%%%%%%%%%%%%%%%%%%%%%%%%%%%%%%%%%%%%%

%----------------------------------------------------------------------------------------
%   PACKAGES AND OTHER DOCUMENT CONFIGURATIONS
%----------------------------------------------------------------------------------------

\documentclass{resume} % Use the custom resume.cls style

\usepackage[left=0.75in,top=0.6in,right=0.75in,bottom=0.6in]{geometry} % Document margins

\name{Liana L. Lo}
\address{461 2nd Ave. \\ San Francisco, CA 94118}
\address{liana.lixia@gmail.com \\ 415-763-7386 \\ linkedin.com/in/lianalo}

\begin{document}

%----------------------------------------------------------------------------------------
%   SUMMARY SECTION
%----------------------------------------------------------------------------------------

\begin{rSection}{Summary}
% % Liana is a full-stack software engineer who loves solving the puzzles that come with her daily work.
% % Liana is a full-stack software engineer who loves working towards a more open and connected world.

Liana is a self-taught software engineer who loves working towards a more open and connected world. Born and raised in San Francisco, her desire to connect with people around the globe brought her to live, study, and work in Phoenix, Beijing, Taipei, Jakarta, and Budapest.

% Liana always develops software with the end user in mind. Born and raised in San Francisco, she has since lived in Phoenix, Beijing, Taipei, Jakarta, and Budapest. Liana applies her global perspective while creating products for an inclusive, international market.

\end{rSection}

%----------------------------------------------------------------------------------------
%   WORK EXPERIENCE SECTION
%----------------------------------------------------------------------------------------

\begin{rSection}{Industry Experience}

\begin{rSubsection}{Prezi}{March 2015 - Present}{Full-stack Software Developer in Core Product}{Budapest, Hungary | San Francisco, CA}
\item ``Prezi for Teams'' is the enterprise platform for our cloud-based (SaaS) presentation software.
% \item Working alongside a product owner, UX researcher, designer, data analyst, and three other engineers on sign-up flows, admin UI, and group payments (Braintree) with Django and Backbone.js in a microservice architecture glued together with Apache Thrift.
\item Working alongside a product owner, UX researcher, designer, and data analyst on sign-up flows and admin UI with Django and Backbone.js in a microservice architecture.
\item Initialized new microservice for email verification, including creating AWS EC2 instances, ELB, MySQL RDS, and setting up automated build-and-deploy pipelines.
\item Internationalization(i18n) for nine languages, A/B testing, integrated logging, monitoring, alerting.
% \item Occasional debugging on pre-production and live production nodes.
% \item Voluntarily participate in user interviews with our UX researcher to provide technical insight of product behavior.
\end{rSubsection}

%------------------------------------------------

\begin{rSubsection}{Prezi}{October 2014 - April 2015}{Software Engineer in Infrastructure, Internal Tools}{Budapest, Hungary}
\item Built a front end UI in AngularJS for Prezi's monitoring and alerting system. Wrote unit and integration tests with Jasmine, Karma, and Protractor. Automated development tasks with Grunt.
% \item Extended Prezi's monitoring API (written in Haskell) with a tagging feature for alerting configurations. Originally, alerting items could be searched only by name substring matching. The new feature allows developers to tag their items with an identifier, which in turn allows groupings of several related alerting items displayed in a single page view for easy analysis.
\item Extended Prezi's monitoring API (written in Haskell) with a tagging feature for alerting configurations.
\end{rSubsection}

%------------------------------------------------

\begin{rSubsection}{Intel Corporation}{2011, 2012 summers; August 2013 - January 2014}{Quality Engineer}{Chandler, AZ}
\item Eliminated microprocessor defect sources in high-volume manufacturing environment.
% \item Continuously drove process improvements to reduce operating costs and maximize factory yield.
\end{rSubsection}

\end{rSection}

%----------------------------------------------------------------------------------------
%   PERSONAL PROJECTS SECTION
%----------------------------------------------------------------------------------------

\begin{rSection}{Personal Projects}

%------------------------------------------------

\begin{rSubsection}{Remote Screen Controller}{github.com/lolilo/screen-controller}{}{}
\item Program enables user to remotely manage content displayed on screens connected within a local network. One master controls slave devices (Raspberry Pi or Apple computers) connected to monitors.
\item Practiced concepts in networking, cross-platform compatibility (Linux, OS X), and security (proxy, iptables, auth headers). Primarily written in Go.
\end{rSubsection}

%------------------------------------------------

\begin{rSubsection}{VisuaLisPy}{github.com/lolilo/VisuaLisPy}{}{}
\item Built a Scheme-to-JavaScript compiler and a JavaScript parser using Python.
\item Constructed browser-based abstract syntax tree visualizer using JavaScript, jQuery, Ajax, JSON, D3.js, HTML, CSS; deployed to Heroku.
\item Set up PostgreSQL database using SQLAlchemy containing example input for visualizer.
\end{rSubsection}

\end{rSection}

% ----------------------------------------------------------------------------------------
%   SKILLS SECTION
% ----------------------------------------------------------------------------------------

\begin{rSection}{Skills and Tools}

% Python, Go, JavaScript, HTML, CSS \enspace $\diamond$ \enspace XML, JSON, REST \enspace $\diamond$ \enspace Protractor, Cucumber \\
Python, Go, JavaScript, HTML, CSS \enspace $\diamond$ \enspace Flask, Django, Cucumber \\
MySQL, PostgreSQL, SQLite \enspace $\diamond$ \enspace Git, Vim, Sublime, IntelliJ \enspace $\diamond$ \enspace AWS, Chef, Jenkins \smallskip

{\em Education on second page}

% \begin{tabular}{ @{} >{\bfseries}l @{\hspace{6ex}} l }
% Computer Languages & Python, Go, JavaScript, HTML, CSS \\
% Protocols \& APIs & XML, JSON, REST \\
% Databases & MySQL, PostgreSQL, SQLite \\
% Testing Frameworks & Protractor, Cucumber \\
% Development Tools & Git, Vim, Sublime, IntelliJ \\
% Infrastructure & AWS, Chef, Jenkins
% \end{tabular}

% {\em Education on second page}

\end{rSection}

%----------------------------------------------------------------------------------------
%   EDUCATION SECTION
%----------------------------------------------------------------------------------------

\begin{rSection}{Education}

{\bf University of California, Berkeley} \hfill {\em Berkeley, CA} \\
B.S. in Chemical Engineering, June 2013 \\
Minor in Indistural Engineering and Operations Research \smallskip

{\bf Peking University}, University of California Education Abroad Program \hfill {\em Beijing, China} \\
Intensive Chinese Program, February 2013 - June 2013 \smallskip

{\bf Hackbright Academy}, Software Engineering Fellowship \hfill {\em San Francisco, CA} \\
Computer Science, February 2014 - May 2014 \smallskip

{\bf National Taiwan Normal University}, Overseas Community Affairs Council \hfill {\em Taipei, Taiwan} \\
Chinese Language and Culture, June 2014 - August 2014 \smallskip \\

\end{rSection}


%----------------------------------------------------------------------------------------
%   EXAMPLE SECTION
%----------------------------------------------------------------------------------------

%\begin{rSection}{Section Name}

%Section content\ldots

%\end{rSection}

%----------------------------------------------------------------------------------------

\end{document}